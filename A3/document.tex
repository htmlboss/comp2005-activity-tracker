
\documentclass{article}

\font\bfbig cmbx12 at 18pt

\usepackage{rotating, amsthm, listings, float}
\lstset{
	breaklines=true,
	columns=flexible,
	basicstyle=\tt}

\theoremstyle{definition}
\newtheorem{query}{Sample Query}

\begin{document}

{\bf\parindent=0pt

	Memorial University of Newfoundland \hfill Fall Semester 2018 
	
	Computer Science 2005 \hfill Group 12

}

\vskip 2pc

{\bfbig

	\hfill Assignment 3 \hfill
}

\vskip 1pc

This assignment instructs us to ``prioritize features of the system you are supposed to implement.'' As the application being developed is an activity logger app, we believe two of the most crucial parts of the app --- its essence --- are the user interface and the backend data storage and retrieval. Because of this, in this assignment to ``create architecturally significant design,'' we have focused on developing the Java-based UI and engineering an efficient database structure for the backend that will enable easy retrieval of large data sets for a particular user. 

We have decided to use a relational database and are using SQLite to implement this. SQLite was chosen for developmental simplicity and availability of a JDBC-SQLite driver.

A prototype of the relational database schema is provided in the tables below, with some sample data.

We also include some sample queries that we may run for various parts of the app.

\begin{table}[H]
	\centering
	\tabcolsep=.5pc
	{\bf Users}
	
	\begin{tabular}{clllc} \hline 
		ID & EmailAddress & Name & DOB & Sex \\ \hline  
		1 & jdoe@acme.com & Johnathan Doe & 23-01-1992 & M \\
		2 & goulding@contoso.com & Annalise Goulding & 03-08-1947 & F \\
		3 & neil@google.com & Neil Flemming & 21-11-1989 & M \\
		4 & avery.smith@mac.com & Avery Smith & 12-12-1973 & M \\
		\hline 
	\end{tabular}
	\caption{\label{tab:users}
		The {\tt User} table will keep track of what users we have in our system. The {\tt ID} column is the primary key and will contain unique values for each user. This key will be used to relate rows in this table to rows in other tables.	
	}
\end{table}

\begin{table}[H]
	\centering
	\tabcolsep=.5pc
	{\bf Passwords}
	
	\begin{tabular}{ccll} \hline 
		ID & UserID & PassHash & DateChanged \\ \hline  
		1 & 1 & 4bd44ba402231f35390c1ae3b76f0154 & 11-10-2017 13:17:08 \\
		2 & 1 & 7f61f924bcea42434fee55f3bdd05c5d & 11-10-2017 14:01:56 \\
		4 & 3 & 18f8caa5842ff58d4a40701a1d377966 & 12-01-2018 07:23:45 \\
		3 & 2 & 634a3583084e873ccf31d58622707219 & 18-02-2018 08:32:45 \\
		5 & 3 & 75c625509efe6deca63084f971067634 & 19-10-2018 17:45:45 \\
		\hline 
	\end{tabular}
	\caption{\label{tab:passwd}
		We store the passwords in a separate table from the users so that we can maintain a password history of arbitrary length and can therefore enforce password history requirements when users change passwords. Again, the {\tt ID} column is the primary key. The password is associated with a user by the second column, {\tt UserID}. This approach also helps users remember their password, {\it f.e.}, if they type the old password, we can match that and remind them that they changed it past week.
	}
\end{table}


\begin{table}[H]
	\centering
	\tabcolsep=.5pc
	{\bf Workouts}
	
	\begin{tabular}{ccllll} \hline 
		ID & UserID & WOType & Date & Duration & kCal \\ \hline 
		1 & 1 & Run & 18-04-2018 & 1:37:04.60 & 1,107 \\  
		2 & 2 & Weights & 01-11-2018 & 1:59:36.78 & 3,114 \\ \hline 
		\hline 
	\end{tabular}
	\caption{\label{tab:workout}
		We log workouts to this table. As before, {\tt ID} is the primary key and rows are associated with users by the {\tt UserID} column. Entries in this table may be updated through the user interface if a user would like to correct data provided by a device's sensor, or if data is being provided to the app incrementally during a workout rather than in a lump-sum import.
	}
\end{table}

\begin{table}[H]
	\centering
	\tabcolsep=.5pc
	{\bf Friends}
	
	\begin{tabular}{cccll} \hline 
		ID & Sender & Receiver & SentDate & ConfirmedDate \\ \hline 
		1 & 1 & 2 & 18-02-2018 07:43.13.68 & 18-02-2018 07:45:12.34 \\
		4 & 3 & 4 & 12-08-2018 19:23:17.32 & {\tt NULL} \\
		\hline 
	\end{tabular}
	\caption{\label{tab:friends}
		This table keeps track of friends and pending friend requests. Confirmed friend requests will have a confirmation date not equal to {\tt NULL}. A confirmation date equal to {\tt NULL} indicates a friend request that has been sent but not accepted or declined yet. If a user declines a friend request or deletes a friendship, the corresponding row will be removed from the table. Here, rows 2 and 3 have been removed, indicating cancelled requests or terminated friendships.
	}
\end{table}

\begin{sidewaystable}[h]
	\centering
	\tabcolsep=.5pc
	{\bf Devices}
	
	\begin{tabular}{ccllll} \hline 
		ID & UserID & MACAddress & DeviceName & DeviceModel & SerialNo \\ \hline  
		1 & 2 & 3d344f1294ca & Annalise's Watch & Apple Watch Series 2 & K5J2WD4K \\
		2 & 1 & 56dc8a3283c4 & Johnathan Doe's FitBit & FitBit Charge 2 & H45DC2456S \\
		3 & 1 & 24c1c9ada464 & John's Apple Watch & Apple Watch Series 3 & KW5DC2NW \\ 
		\hline 
	\end{tabular}
	\caption{\label{tab:device}
		Entries in this table are associated with a user's device(s). Here we record things like hardware MAC address, device model, device name. As before the {\tt ID} column is the primary key for the table and {\tt UserID} is how a device is associated with a user. 
	}
\end{sidewaystable}



\begin{sidewaystable}[h]
	\centering
	\tabcolsep=.5pc
	{\bf SocialStatus}
	
	\begin{tabular}{ccp{3in}l} \hline 
		ID & UserID & Status & PostDate \\ \hline  
		1 & 2 & Had a great run along the lakeshore this morning and saw the most beautiful sunrise! & 18-08-2017 09:12:09.98 \\
		2 & 4 & First time in the pool this month; out of shape! & 01-11-2018 13:04:12.92 \\ 
		3 & 2 & New 1RM in the gym today!! & 01-11-2018 14:48:32.12 \\
		\hline 
	\end{tabular}
	\caption{\label{tab:social}
		This table records optional status updates that users may post for their friends to see. Typically these will be about their workouts. Columns {\tt ID} and {\tt UserID} are as before.
	}
\end{sidewaystable}



\begin{query}
	We want to get all passwords for the user with email address neil@google.com.
	\begin{lstlisting}[gobble=8]
	SELECT Passwords.PassHash, Password.DateChanged from Users, Passwords WHERE Users.EmailAddress = 'neil@google.com';
	\end{lstlisting}
	This returns
	\begin{center}
		\begin{tabular}{ll} \hline 
			PassHash & DateChanged \\ \hline 
			18f8caa5842ff58d4a40701a1d377966 & 12-01-2018 07:23:45 \\
			75c625509efe6deca63084f971067634 & 19-10-2018 17:45:45 \\ 
			\hline 
		\end{tabular}
	\end{center}
\end{query}

\begin{query}
	We want the most recent social post for Annalise Goulding. We will definitely have this user's ID on hand as this is how we identify and keep track of users for everything. So we then query the following.
	\begin{lstlisting}[gobble=8]
	SELECT Status FROM SocialStatus WHERE UserID = 2 ORDER BY PostDate DESC LIMIT 1;
	\end{lstlisting}
	This returns the posting from 01-11-2018 14:48:32.12, ``New 1RM in the gym today!!''
\end{query}



\end{document}